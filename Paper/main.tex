\documentclass[modern]{aastex631}
\usepackage{amsmath}
\usepackage{soul}

% Abundance and stellar macros
\newcommand{\mgfe}[0]{[{\rm Mg/Fe}]} 
\newcommand{\Acc}{A_{\rm cc}}
\newcommand{\AIa}{A_{\rm Ia}} 
\newcommand{\RIa}{R_{\rm Ia}^X}
\newcommand{\aIa}{\alpha_{\rm Ia}}
\newcommand{\acc}{\alpha_{\rm cc}}
\newcommand{\afe}[0]{[\alpha/{\rm Fe}]}
\newcommand{\femg}{[{\rm Fe}/{\rm Mg}]} 
\newcommand{\xmg}{[{\rm X}/{\rm Mg}]} 
\newcommand{\mgh}{[{\rm Mg}/{\rm H}]}
\newcommand{\feh}[0]{[{\rm Fe/H}]} 
\newcommand{\pIa}{p_{\rm Ia}^{\rm X}}
\newcommand{\pcc}{p_{\rm cc}^{\rm X}}
\newcommand{\fcc}{f_{\rm cc}}
\newcommand{\xfe}{[{\rm X}/{\rm Fe}]} 
\newcommand{\pIasun}{p_{\rm Ia, \odot}^{\rm X}}
\newcommand{\pccsun}{p_{\rm cc, \odot}^{\rm X}}
\newcommand{\logg}{\log(g)}
\newcommand{\teff}{T_{\rm eff}}
\newcommand{\kpc}{\rm \; kpc}
\newcommand{\kel}{\rm \; K}
\newcommand{\msun}{M_{\odot}}
\newcommand{\rgc}{R_{\rm GC}}
\newcommand{\msunvice}{M_{\odot}/M_{\odot \rm formed}}
\newcommand{\MZAMS}{M_{\rm ZAMS}}
\newcommand{\Msun}{M_{\odot}}
\newcommand{\qIax}{q_{\rm Ia}^{\rm X}}
\newcommand{\qccx}{q_{\rm cc}^{\rm X}}
\newcommand{\qagbx}{q_{\rm AGB}^{\rm X}}
\newcommand{\qIa}{q_{\rm Ia}}
\newcommand{\qcc}{q_{\rm cc}}
\newcommand{\qagb}{q_{\rm AGB}}
\newcommand{\Aagb}{A_{\rm AGB}}
\newcommand{\eumg}{[{\rm Eu}/{\rm Mg}]} 
\newcommand{\mgmn}{[{\rm Mg}/{\rm Mn}]} 
\newcommand{\alfe}{[{\rm Al}/{\rm Fe}]} 
\newcommand{\omg}{[{\rm O}/{\rm Mg}]} 
\newcommand{\xo}{[{\rm X}/{\rm O}]} 
\newcommand{\xh}{[{\rm X}/{\rm H}]} 
\newcommand{\angstrom}{\textup{\AA}}
\newcommand{\ejg}[1]{\textcolor{red}{EJG: #1}}

%% Reintroduced the \received and \accepted commands from AASTeX v5.2
%\received{March 1, 2021}
%\revised{April 1, 2021}
%\accepted{\today}

%\submitjournal{ApJ}

\newcommand{\name}{\textsl{Foo}} % Hogg's just making this up; change it!!
\newcommand{\documentname}{\textsl{Article}}

\shorttitle{data-driven few-process model for nucleosynthesis}
\shortauthors{griffith and hogg}
\addtolength{\topmargin}{-0.4in} % trust in Hogg
\addtolength{\textheight}{0.7in} % trust in Hogg
\setlength{\parindent}{1.6em}
\renewcommand{\paragraph}[1]{\bigskip\par\noindent{\textbf{#1}}~---}
\sloppy\sloppypar\raggedbottom\frenchspacing % Trust Hogg

\graphicspath{{./}{Figures/}}
\begin{document}

\title{\name: A data-driven few-process model for nucleosynthesis}

\correspondingauthor{Emily J. Griffith}
\email{Emily.Griffith-1@colorado.edu}

\author[0000-0001-9345-9977]{Emily J. Griffith}
\altaffiliation{NSF Astronomy and Astrophysics Postdoctoral Fellow}
\affiliation{Center for Astrophysics and Space Astronomy, Department of Astrophysical and Planetary Sciences, University of Colorado, 389~UCB, Boulder,~CO 80309-0389, USA}

\author[0000-0003-2866-9403]{David W. Hogg}
\affiliation{Center for Cosmology and Particle Physics, Department of Physics, New York University, 726~Broadway, New~York,~NY 10003, USA}
\affiliation{Max-Planck-Institut f{\"u}r Astronomie, K{\"o}nigstuhl 17, D-69117 Heidelberg, Germany}
\affiliation{Flatiron Institute, 162 Fifth Avenue, New~York,~NY 10010, USA}

% \author[0000-0001-7775-7261]{David H. Weinberg}
% \affiliation{The Department of Astronomy and Center of Cosmology and AstroParticle Physics, The Ohio State University, Columbus, OH 43210, USA}
% \affiliation{The Institute for Advanced Study, Princeton, NJ, 08540, USA}

\begin{abstract}\noindent % trust Hogg
Stellar surface abundances look like they are produced by two dominant processes, one prompt and one delayed.
We analyze YYY abundance ratios (ratioed to Mg) for XXXX stars from APOGEE DR17 with a flexible two-process model, in which all element process amplitudes are free parameters.
The two-process model describes each element abundance of each star as the vector sum of a CCSN and SNIa process.
Prior work derives process amplitudes from median observed high-Ia and low-Ia abundance trends; we relax these assumptions and simultaneously fit all model parameters to the data, with only minimal constraints to keep the processes interpretable.
We compare fit parameters from this new method to those derived from prior work [that the two methods produce similar results but with XXX differences and identify similar outlier stars??].
Our flexible method---dubbed \name---tests the validity of the standard two-process assumptions, such as Mg being a pure CCSN element, and that Solar abundances are generated by a roughly half-half combination of CCSN and SNIa processes.
We find [that the assumptions of prior work are generally confirmed??].
We capitalize on the flexibility to explore the addition of a third process, agnostic about its origin.
The third process most improves [X elements? most significant for X elements?].
Most importantly, \name{} can be extended to populations at very low metallicity, and populations without two clear abundance sequences, such as the LMC/SMC, Sagittarius, and Gaia-Enceladus (??), making it a general diagnostic for enrichment; in principle, it can also be used to constrain fundamental nucleosynthetic parameters.
\end{abstract}

\keywords{foo --- bar}

\section*{}\clearpage
HOGG SAY: Here are some things to do or think about:
\begin{itemize}
  \item Should we think of this as a 2-process model or as a $K$-process model, where $K$ is often 2? I'm leaning towards $K$.
  \item \ejg{I like the K-process model. (This is named $N$-Process model in W22)}
  \item What are we going to do in terms of interpreting the results? These process vectors aren't exactly yields; they are something way more complicated, right?
  \item \ejg{Interpret in terms of $\fcc$. Can also make inferences about the metalicity dependence of SNIa/CCSN enrichment to given elements based on how $\qcc$ and $\qIa$ evolve with time.}
\end{itemize}

\section{Introduction}\label{sec:intro}

After hydrogen, helium, lithium, and beryllium, essentially all other naturally occurring elements are made in stars, and the collisions of stars.
We are literally made of star stuff.
Stellar surface abundances---the abundances measured by taking a spectrum of a stellar photosphere---are thought to deliver a relatively unprocessed (for most elements) record of the element abundances in the gas from which the star formed (though see \ejg{Add citations}).
These birth abundances were set by a combination of nucleosynthetic processes involved in making heavy atomic nuclei, and astrophysical processes involved in delivering atoms from stellar interiors to star-formation sites \citep[e.g.,][]{johnsonja2020}.
Thus nuclear physics and a wide swath of astrophysics are critically intertwined in our understanding of stellar surface abundances.
That's important, and motivates a lot of theoretical, experimental, and observational work in nuclear physics and astrophysics.

At the present day, stellar surface abundances are not very well explained by \textsl{ab initio}, physics-driven models.
Yields vary from set to set, as they are dependent on progenitor properties and explosion assumptions \citep[e.g.,][]{rybizki2017, griffith2021b}. 
The wide parameter space of progenitor and supernovae models coupled with uncertainties in reaction rates and explosion physics hinder the creation of an accurate nucleosynthetic model from theory alone.
In the long run, it is incumbent upon us to understand these issues and correct the assumptions or calculations underlying our nucleosynthetic and astrophysical models.
In the short run, however, we gather data---tens of millions of abundance measurements on millions of stars in different astronomical surveys such as RAVE, SEGUE, LAMOST, Gaia-ESO, APOGEE, GALAH, and H3 \citep{steinmetz2006, yanny2009, gilmore2012, desilva2015, luo2015, majewski2017, conroy2019}.
This raises the question: \emph{Can we take a data-driven approach to nucleosynthesis?}

In this \documentname{}, we build a purely data-driven model for the surface element abundances observed in stars.
We treat each star as being a linear combination of two nucleosynthetic processes, one of which is primarily responsible for the $\alpha$-element Mg \citep[core collapse supernovae (CCSN), e.g.][]{andrews2017}, and one of which is primarily \emph{not} (Type-Ia supernovae (SNIa)).
Beyond that up-front assumption, we try to be agnostic about how the elements are produced.

We build upon the work of \citet[][hereafter G22]{griffith2019, griffith2022} and \citet[][hereafter W22]{weinberg2019, weinberg2022}, who used the bimodality in [Mg/Fe] vs. [Fe/H] \citep[e.g.,][]{fuhrmann1998, bensby2003, adibekyan2012} to separate stars into populations with high and low SNIa enrichment. Using the median [X/Mg] vs. [Mg/H] abundance trends, these works explain data from the \textsl{GALAH}\footnote{GALAH = GALactic Archaeology with HERMES.} and SDSS-IV \textsl{APOGEE}\footnote{APOGEE = Apache Point Observatory Galactic Evolution Experiment, part of the Sloan Digital Sky Survey} surveys, respectively, with a two-process model. Because the median abundance trends in [X/Mg] vs. [Mg/H] space are largely insensitive to aspects of chemical evolution, such as outflows and variations in star formation history \citep{weinberg2019}, the population abundance trends are set by the nucleosynthetic processes and can be used to empirically constrain Galactic enrichment. To fit the two-process model to survey data, G22 and W22 assume that Mg is purely produced by CCSN, Fe is produced in equal amounts by CCSN and SNIa, and that the CCSN/SNIa yields of Mg and Fe are metallicity independent. While the first assumption is firmly grounded in nucleosynthetic theory \citep[e.g.,][]{andrews2017, rybizki2017}, the others are based on APOGEE abundance patterns and may not be realistic constraints.

Beyond the two-process model, many elements have contributions from additional nucleosynthetic processes, such as the rapid ($r$) and slow ($s$) neutron capture processes \citep{arlandini1999, bisterzo2014} in asymptotic giant branch (AGB) stars \citep[e.g.,][]{simmerer2004, karakas2016} and merging neutron stars \citep{kilpatrick2017}, or atypical supernovae explosion. After predicting stellar abundances from $\feh$ and $\mgfe$, \citet{ting2022} identify correlated abundance residuals that are unexplained by observational uncertainties, indicative of additional nucleosynthetic processes that standard disk CCSN and SNIa enrichment cannot explain. Results from G22 and W22 support this conclusion, and both works attempt to add additional processes to their models to account for non-CCSN and non-SNIa enrichment, though in a very restrictive manner.

\ejg{Add paragraph about how little work has been done to optimize stellar abundances to solve for enrichment processes in a more data-driven way. \citet{casey2019} apply a clustering algorithm to GALAH DR2 data, identifying latent factors corresponding to 6 processes. Difficult to interpret and breaks down with more data.}

In this work, our main innovations are to relax the assumptions made in G22 and W22, to be more agnostic about the nucleosynthetic processes, and to be more principled with the measurements or inferences from data. The benefit we will gain from this is better performance at fitting the data.
The cost we might pay is a possible reduction in interpretability.
We'll come back to all that at the end.

\ejg{Placeholder. We attempt to find the intersection between reliable facts about nucleosynthesis and good abundance measurements. We are trying to build an edifice of nucleosynthesis on a small number of facts. Because Mg is well-measured and well-understood theoretically, it's a good foundation for our model. We have a degeneracy with Fe because we don't have a nucleosynthesis fact.}

In Section~\ref{sec:data} we describe the APOGEE data samples fit in this paper. In Section~\ref{sec:model} we outline our $K$-process model, the assumptions we make, and our implementation of the optimization routine. We apply the data-driven $K$-process model to the W22 APOGEE sample and present our findings in Section~\ref{sec:results_W22}. \ejg{Add ending discussion sections...}


\section{Data Samples}\label{sec:data}

In this paper, we employ stellar abundances from APOGEE DR17 \citep{abdurrouf2022}, part of SDSS-IV \citep{majewski2017}. The APOGEE survey obtains high-resolution ($R\sim22,500$) near-infrared (IR) observations \citep{wilson2019} for stars in the Galactic disk, halo, bulge, and nearby satellites/streams. Observations are taken with two nearly identical spectrographs on the 2.5m Sloan Foundation telescope \citep{wilson2019} at Apache Point Observatory in New Mexico and the 2.5m du Pont Telescope \citep{bowen1973} at the Las Companas Observatory in Chile. Spectral data are reduced and calibrated with the APOGEE data processing pipeline \citep{nidever2015}, after which stellar parameters and parameters are calculated with ASPCAP \citep[APOGEE Stellar Parameter and Chemical Abundance Pipeline][]{holtzman2015, garcia2016}. See \citet[][DR16]{jonsson2020} and Holtzman et al. (in prep., DR17) for a more detailed description of APOGEE data reduction and analysis, and \citep{zasowski2013, zasowski2017, santana2021} for a discussion of survey targeting.
\ejg{Is the DR17 paper out yet?}

APOGEE DR17 reports stellar parameters including $\teff$ and $\logg$ as well as 20 elemental abundances: C, \ion{C}{1}, N, O, Na, Mg, Al, Si, S, K, Ca, Ti, \ion{Ti}{2}, V, Cr, Mn, Fe, Co, Ni, and Ce for 657,135 stars. Among these elements and ions, some are measured more precisely than others. We exclude Ti from our analysis as there are large differences between the abundances derived from the \ion{Ti}{1} and \ion{Ti}{2} lines \citep{jonsson2020}. We also exclude P, as the P abundances are measured from a few very weak spectra features, and the abundance distribution displays strong artifacts \citep{jonsson2020}. Among the remaining elements, we note the following concerns:
\begin{itemize}
\itemsep0em
    \item Na - Spectral features are weak, one of the least precise elements in APOGEE (but matches optical data pretty well)
    \item S - Large abundance scatter in observations and no verification with optical data
    \item Cr - Significant systematic artifacts at super solar metallicity \citep{griffith2021a}
    \item Mn - Strong NLTE effects exist that are not taken into account
    \item Co - Abundance derived from one line, causing larger abundance scatter
    \item Ce - Abundance derived from one line. DR17 is the first time Ce abundances have been included in APOGEE data products. 
    \item Al,  V, and Mn require large offsets to place solar metallicity stars at solar [X/Fe].
\end{itemize}
\ejg{Will want to condense list for paper. Add a short description of some abundance systematics that we see re \citet{jonsson2020} and \citet{griffith2021a}}.

In this paper, we will explore two subsamples of APOGEE DR17 data: (1) the W22 sample, optimized to minimize statistical and systematic errors while probing a large portion of the Galactic disk, and (2) a sample with less stringent cuts, but better coverage of the low-metallicity ([Fe/H] $< -0.5$) disk.

\subsection{The W22 Sample}

W22 select a subset of APOGEE DR17 stars with the goal of minimizing statistical errors from poor observations and systematic errors from abundance trends with $\teff$ and/or $\logg$ while preserving a sufficient number of stars to conduct a meaningful statistical analysis across the Galactic disk. To remove poor quality data points, W22 require ASPCAP flags \texttt{STAR\_BAD} \texttt{NO\_ASPCAP\_RESULT} equal zero and remove all stars with \texttt{FE\_H\_FLAG} or \texttt{MG\_FE\_FLAG} set. They only include stars from the main survey sample (\texttt{EXTRATARG} = 0), and use named abundances (\texttt{X\_FE}), as recommended by \citet{jonsson2020}. In addition to these quality cute, W22 apply the following sample selection cuts:
\begin{itemize}
\itemsep0em
    \item $R=3-13$ kpc, $|Z| \leq 2$ kpc (distances from \citet{leung2019})
    \item $-0.75 \leq \mgh \leq 0.45$
    \item S/N $\geq 200$ for $\mgh > -0.5$ and S/N $\geq 100$ for $\mgh < -0.5$
    \item $\logg = 1-2.5$ dex
    \item $\teff = 4000-4600$ K
\end{itemize}
By selecting stars from a small range of $\logg$ and $\teff$, they eliminate the red clump stars and minimize stellar parameter-dependent abundance trends \citep[see][]{griffith2021a}. The cuts result in a sample of 34,410 stars. 

W22 present abundance for Mg, O, Si, S, Ca, C+N, Na, Al, K, Cr, Fe, Ni, V, Mn, Co, and Ce. While the surface abundances of C and N differ from the stellar birth abundances for RGB stars due to the CNO processes and dredge up events \citep{iben1965, shetrone2019}, the total C+N abundance remains constant. W22 consider C+N as an element in their analysis, taking [(C+N)/H] to be 
\begin{equation}
    [\text{C+N}/\text{H}] = \log_{10}(10^{\text{[C/H]}+8.39} + 10^{\text{[N/H]}+7.78}) - \log_{10}(10^{8.39} + 10^{7.78})
\end{equation}
using logarithmic solar abundances for C (8.39) and N (7.78) from \citet{grevesse2007}.

\ejg{W22 use X/Fe error from APOGEE and don't convert it to X/Mg....Use C/Fe error for the C+N value because the ratio is mostly C... Need error discussion here and in next section.}

In the analysis of each element, X, stars with \texttt{X\_FE\_FLAG} are dropped (maximum of $\sim 560$ for Ce). We plot the distribution of abundances in [X/Mg] vs. [Mg/H] in Figure~\ref{fig:w22_xmg}. All elemental abundances have zero-point offsets applied and have been de-trended for correlations with $\teff$, as in W22 (see their Table 1). 

\begin{figure}[htb!]
    \centering
    \includegraphics[width=\textwidth]{Figures/W22_xmg.pdf}
    \caption{W22 stellar abundance distributions. The number in the top right corner indicates the number of stars used in the analysis of that element. Elements are ordered by atomic number.}
    \label{fig:w22_xmg}
\end{figure}

\subsection{Our Sample}

\ejg{Not sure if this should go here or after we discuss the success of the K-process model on W22 data...}

While the W22 sample presented above is a good first test case of \name, we are interested in the behavior of our model when applied to a wider sample of data, in particular extending the metalicity range to lower values of [Mg/H]. To construct this sample, we apply the same quality cuts as taken by W22 (\texttt{STAR\_BAD} = 0, \texttt{NO\_ASPCAP\_RESULT} = 0, \texttt{FE\_H\_FLAG} = 0, \texttt{MG\_FE\_FLAG} = 0, and (\texttt{EXTRATARG} = 0). We add a cut for stars with the \texttt{ROTATION\_WARNING} flag set. We take the following sample selection cuts: 
\itemsep0em
\begin{itemize}
    \item S/N $\geq 100$
    \item $0 < \logg < 2.5$ dex
    \item $\teff = 4000-4600$ K
\end{itemize}
We make no cuts on the Galactic location, including stars from across the Galactic disk and halo, and make no cuts on the [Mg/H] values of our sample. The wider $\logg$ ranges expand the sample of giants. We select stars in the same small $\teff$ range as W22 to exclude the red clump \citep{vincenzo2021} and cool stars that show abundance artifacts \citep{jonsson2020}.
Our final sample includes 51,386 stars and extends to [Mg/H]=-2.2. As in W22, we only analyze [X/Mg] abundances for stars where the [X/Fe] value is unflagged. This has a minimal effect for most elements but reduces the sample size to 49,851 for Ce. 

\ejg{Plus a magnitude cut as a proxy for distance? Right now I have H < 12, but I don't think that the cut impacts the global trends much...}

\begin{figure}[htb!]
    \centering
    \includegraphics[width=\textwidth]{Figures/xmg.pdf}
    \caption{Stellar abundance distributions for our wider stellar sample. The number in the top right corner indicates the number of stars used in the analysis of that element. Elements are ordered by atomic number.}
    \label{fig:w22_xmg}
\end{figure}

The expansion of the data sample does introduce abundance systematics that the W22 selection sought to exclude. As discussed in \citet{jonsson2020} and \citet{griffith2021a}, APGOEE DR16 and DR17 suffer from poorly understood abundance artifacts, such as a ``finger'' in [$\alpha$/M] and low-[X/Fe] banding in elements such as Al and Ni. In our expanded data set, we observe potential abundance artifacts in Al, K, V, and Cr. \ejg{Add more here. Stricter temperature cut helped to remove the artifacts...}

\ejg{Should check the spatial distribution of the final sample. Need to decide if we will also apply zero point offsets}

\ejg{Also need to add discussion that almost all stars below [Mg/H] of -0.75 appear chemically accreted...}

\section{Methods and Models}\label{sec:model}

We model each star as the sum of a CCSN and SNIa process vector (constant for all elements) and CCSN/SNIa amplitude such that 

\begin{equation}
    m_{ij} = \log_{10}(A_i^{\rm CC} \vec{q}_{{\rm CC}, j}^{\,Z} + A_i^{\rm Ia} \vec{q}_{{\rm Ia}, j}^{\,Z}) 
\end{equation}

where the observed value of $\xh$ can be described as the model plus observational noise and/or contributions from an additional non-CCSN/SNIa process:

\begin{equation}
    \xh_{ij} = m_{ij} + \text{noise}.
\end{equation}

We fit this model to the data with a chi-squared ($\chi^2$) minimization such that 

\begin{equation}
    \chi^2 = \sum_{i=1}^{N} (\xh_{ij} - m_{ij})^{\rm T} c_j^{-1} (\xh_{ij} - m_{ij}) + \text{regularization}~,
\end{equation}
where $c_j^{-1}$ is a matrix of the inverse variance squared,
\begin{equation}
    c^{-1} = \begin{bmatrix}
\frac{1}{\sigma_j^2} & 0 & ...\\
0 & \frac{1}{\sigma_j^2} & ... \\
0 & 0 & ... 
\end{bmatrix}
\end{equation}
$N$ stars, such that $1 \leq i \leq N$, and ``regularization'' stands in for some additional penalization terms that we will describe below.

$M$ elements, such that $1 \leq j \leq M$ where $j \in \{ \text{Mg, O, Si, S, Ca, CN, Na, Al, K, Cr, Fe, Ni, V, Mn, Co, Ce} \}$, the elements in APGOEE DR17 and used in \citet{weinberg2022}.

\subsection{Assumptions}\label{subsec:assumptions}

Some of the assumptions that go into the two-process model \name{} are the following:

\paragraph{Two processes}
HOGG whatever.

\paragraph{Linearity}
At every metallicity, the abundances of a star can be expressed as a linear combination of two processes.
These processes themselves might depend on metallicity, but a linear sum is sufficient to explain all element abundances.
Because of dependences of yields on detailed abundances, and because different stars can get to their metallicities by different histories, this assumption must be slightly wrong in detail.

\paragraph{Non-negativity}
All process amplitudes for all elements are non-negative (elements heavier than XXX are only produced, not ever destroyed, by the two processes),
and all stars are made of a non-negative sum of the two non-negative processes.
This makes the model similar to a non-negative matrix factorization (\citealt{nnmf}).

\paragraph{Magnesium is a core-collapse element only}
All magnesium is produced in the core-collapse (CC) process, and none in the type-Ia (Ia) process.
This breaks (something like) a rotational symmetry in the process space, and makes the processes interpretable in terms of supernova types.

\paragraph{Metallicity dependence}
The fact that all magnesium is produced in the CC process is an assumption that is independent of metallicity.
All other process vectors are permitted to float as a function of metallicity.
The only exceptions are the magnesium level in the CC process and the iron level in the Ia process, but these constraints are just model normalizations; they don't restrict the overall model freedom, which is about abundance ratios.

\paragraph{\textsl{APOGEE} abundances}
We are assuming that the \textsl{APOGEE} abundances can be used for this project.
This is not quite the same as assuming that they are correct.
It is more like assuming that it is possible and useful to build an interpretable 2-process model to explain them.
EMILY: WE NEED A DATA SECTION that explains how we put together the data and which data we used.

\bigskip
These assumptions (above) mirror the assumptions of prior work (\citealt{griffith2022}; \citealt{weinberg2022}) but they are weaker.
In particular, they don't assume anything about the relationships between the processes and the morphologies of observed element-abundance ratio diagrams.
In addition to the above assumptions, which are about nucleosynthesis, there are assumptions about the data:

\paragraph{Measurement uncertainties}
HOGG

\paragraph{Likelihood function}
HOGG

\subsection{Constraints and Regularizations}\label{subsec:regularizations}

Some of the assumptions in our model directly translate to constraints and regularization. 

We always enforce that the process vectors and amplitudes are greater than zero:
\begin{equation}
    \vec{q}_{{\rm CC}, j}^{\,Z}, \, \vec{q}_{{\rm Ia}, j}^{\,Z} \geq 0 \quad \forall \; j
\end{equation}

\begin{equation}
    A_i^{\rm CC}, \, A_i^{\rm Ia} \geq 0 \quad \forall \; i.
\end{equation}

We will start with the constraints from \citet{weinberg2022} as well:

\begin{equation}
    \vec{q}_{{\rm CC, Mg}}^{\,Z_{\odot}} = 1 \quad \vec{q}_{{\rm Ia, Mg}}^{\,Z_{\odot}} = 0 \quad 
    \vec{q}_{{\rm CC, Fe}}^{\,Z_{\odot}} = 0.5 \quad \vec{q}_{{\rm Ia, Fe}}^{\,Z_{\odot}} = 0.5 
\end{equation}

\begin{equation}
    \vec{q}_{{\rm CC, Mg}}^{\,Z} = \vec{q}_{{\rm CC, Mg}}^{\,Z_{\odot}} \quad 
    \vec{q}_{{\rm CC, Fe}}^{\,Z} = \vec{q}_{{\rm CC, Fe}}^{\,Z_{\odot}}  \quad 
    \vec{q}_{{\rm Ia, Fe}}^{\,Z} = \vec{q}_{{\rm Ia, Fe}}^{\,Z_{\odot}} 
\end{equation}

\begin{equation}
    \vec{q}_{{\rm CC}, j}^{\,Z_{\odot}} + \vec{q}_{{\rm Ia}, j}^{\,Z_{\odot}} = 1 \quad \forall \; j
\end{equation}

\section{Experiments and Results}

\section{Discussion}

\begin{itemize}
    \item What is the connection between this model and latent variable models of machine learning
    \item Casual inferences - which abundances are caused by the Mg abundance and which are not (statistician perspective)
    
\end{itemize}

\section{Summary}

\section{Acknowledgements}
It is a pleasure to thank
  Soledad Villar (JHU)
for valuable discussions.
This project benefited enormously from the Python package \texttt{jax} \citep{jax}.
E.J.G. is supported by an NSF Astronomy and Astrophysics Postdoctoral Fellowship under award AST-2202135.

% \software{Matplotlib \citep{hunter2007}, NumPy \citep{harris2020}, Pandas \citep{pandasa, pandasb}, Astropy \citep{astropy2013, astropy2018, astropy2022}, astroNN \citep{bovy2017}, EXOFASTv2 \citep{eastman2013, eastman2019}, iSpec \citep{blanco2014, blanco2019}, MOOG \citep{sneden1973}, galpy \citep{bovy2015, mackereth2018}}, jax \citep{jax}

\bibliography{sample631}{}
\bibliographystyle{aasjournal}

\end{document}

